% Relatório Técnico
\documentclass[12pt,a4paper]{article}

% Pacotes
\usepackage[utf8]{inputenc}
\usepackage[portuguese]{babel}
\usepackage[T1]{fontenc}
\usepackage{graphicx}
\usepackage{listings}
\usepackage{xcolor}
\usepackage{hyperref}

\graphicspath{{imagens/}}

% Configuração de código
\lstset{
  language=Python,
  basicstyle=\ttfamily\small,
  keywordstyle=\color{blue},
  commentstyle=\color{gray},
  stringstyle=\color{red},
  numbers=left,
  numberstyle=\tiny,
  frame=single,
  breaklines=true
}

% Informações
\title{Relatório Técnico}
\author{Equipe de Desenvolvimento}
\date{\today}

\begin{document}

\maketitle

\tableofcontents
\newpage

\section{Sumário Executivo}

Este relatório técnico apresenta os resultados do projeto desenvolvido.

\section{Introdução}

\subsection{Contexto}

Descreva o contexto do projeto.

\subsection{Escopo}

Defina o escopo do trabalho.

\section{Análise de Requisitos}

\subsection{Requisitos Funcionais}

\begin{enumerate}
  \item RF01 - Descrição do requisito funcional 1
  \item RF02 - Descrição do requisito funcional 2
  \item RF03 - Descrição do requisito funcional 3
\end{enumerate}

\subsection{Requisitos Não-Funcionais}

\begin{enumerate}
  \item RNF01 - Desempenho
  \item RNF02 - Segurança
  \item RNF03 - Usabilidade
\end{enumerate}

\section{Arquitetura do Sistema}

Descreva a arquitetura do sistema desenvolvido.

\subsection{Componentes}

\begin{itemize}
  \item \textbf{Frontend:} Interface do usuário
  \item \textbf{Backend:} Lógica de negócio
  \item \textbf{Banco de Dados:} Armazenamento de dados
\end{itemize}

\section{Implementação}

\subsection{Tecnologias Utilizadas}

\begin{table}[h]
  \centering
  \begin{tabular}{|l|l|}
    \hline
    \textbf{Componente} & \textbf{Tecnologia} \\
    \hline
    Frontend & React.js \\
    Backend & Python/Flask \\
    Banco de Dados & PostgreSQL \\
    \hline
  \end{tabular}
  \caption{Stack tecnológico}
  \label{tab:tecnologias}
\end{table}

\subsection{Exemplo de Código}

\begin{lstlisting}[caption=Exemplo de função em Python]
def calcular_media(valores):
    """
    Calcula a média de uma lista de valores.
    """
    if not valores:
        return 0
    return sum(valores) / len(valores)

# Exemplo de uso
notas = [8.5, 9.0, 7.5, 8.0]
media = calcular_media(notas)
print(f"Média: {media}")
\end{lstlisting}

\section{Testes}

\subsection{Testes Unitários}

Descreva os testes unitários realizados.

\subsection{Testes de Integração}

Descreva os testes de integração.

\subsection{Resultados dos Testes}

\begin{table}[h]
  \centering
  \begin{tabular}{|l|c|c|}
    \hline
    \textbf{Tipo de Teste} & \textbf{Total} & \textbf{Aprovados} \\
    \hline
    Unitários & 150 & 148 \\
    Integração & 45 & 44 \\
    Sistema & 20 & 20 \\
    \hline
  \end{tabular}
  \caption{Resultados dos testes}
  \label{tab:testes}
\end{table}

\section{Conclusões}

Apresente as conclusões do projeto.

\subsection{Objetivos Alcançados}

Liste os objetivos alcançados.

\subsection{Trabalhos Futuros}

Sugira melhorias e trabalhos futuros.

\section{Referências}

\begin{itemize}
  \item Documentação oficial das tecnologias utilizadas
  \item Artigos e tutoriais consultados
  \item Normas e padrões seguidos
\end{itemize}

\end{document}


