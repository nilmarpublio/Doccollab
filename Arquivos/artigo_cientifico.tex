% Artigo Científico - Modelo
\documentclass[12pt,a4paper]{article}

% Pacotes
\usepackage[utf8]{inputenc}
\usepackage[portuguese]{babel}
\usepackage[T1]{fontenc}
\usepackage{graphicx}
\usepackage{amsmath}
\usepackage{cite}
\usepackage{hyperref}

\graphicspath{{imagens/}}

% Título e autores
\title{Título do Artigo Científico}
\author{Autor 1\thanks{Instituição 1} \and Autor 2\thanks{Instituição 2}}
\date{\today}

\begin{document}

\maketitle

\begin{abstract}
Este é o resumo do artigo científico. Deve conter uma breve descrição do problema, metodologia, resultados e conclusões. Recomenda-se entre 150 e 250 palavras.

\textbf{Palavras-chave:} LaTeX, Artigo Científico, Pesquisa, Metodologia
\end{abstract}

\section{Introdução}

A introdução apresenta o contexto do trabalho, a motivação da pesquisa e os objetivos do estudo.

\subsection{Motivação}

Descreva aqui a motivação para realizar esta pesquisa.

\subsection{Objetivos}

\begin{itemize}
  \item Objetivo geral
  \item Objetivos específicos
\end{itemize}

\section{Revisão da Literatura}

Apresente aqui a revisão bibliográfica sobre o tema.

\section{Metodologia}

Descreva os métodos utilizados na pesquisa.

\subsection{Materiais}

Liste os materiais utilizados.

\subsection{Procedimentos}

Descreva os procedimentos experimentais.

\section{Resultados}

Apresente os resultados obtidos.

\subsection{Análise Estatística}

Resultados da análise estatística podem ser apresentados em tabelas:

\begin{table}[h]
  \centering
  \begin{tabular}{|l|c|c|c|}
    \hline
    \textbf{Variável} & \textbf{Média} & \textbf{Desvio Padrão} & \textbf{p-valor} \\
    \hline
    Grupo A & 25.3 & 4.2 & 0.032 \\
    Grupo B & 28.7 & 3.8 & 0.018 \\
    \hline
  \end{tabular}
  \caption{Resultados estatísticos}
  \label{tab:resultados}
\end{table}

\section{Discussão}

Discuta os resultados obtidos e compare com a literatura.

\section{Conclusão}

Apresente as conclusões do trabalho e sugestões para trabalhos futuros.

\section{Agradecimentos}

Agradecimentos às instituições e pessoas que contribuíram para o trabalho.

\begin{thebibliography}{9}

\bibitem{exemplo1}
Autor, A. (2024). \textit{Título do Livro}. Editora.

\bibitem{exemplo2}
Autor, B. e Autor, C. (2023). Título do Artigo. \textit{Nome da Revista}, 10(2), 123-145.

\end{thebibliography}

\end{document}


