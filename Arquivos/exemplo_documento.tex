% Documento de Exemplo - DocCollab
\documentclass[12pt,a4paper]{article}

% Pacotes essenciais
\usepackage[utf8]{inputenc}
\usepackage[portuguese]{babel}
\usepackage[T1]{fontenc}
\usepackage{graphicx}
\usepackage{amsmath}
\usepackage{hyperref}

% Configuração de imagens
\graphicspath{{imagens/}}

% Informações do documento
\title{Documento de Exemplo}
\author{DocCollab}
\date{\today}

\begin{document}

\maketitle

\section{Introdução}

Este é um documento de exemplo criado no \textbf{DocCollab}, um editor LaTeX colaborativo online.

\subsection{Recursos do Sistema}

O DocCollab oferece diversos recursos:

\begin{itemize}
  \item Editor LaTeX profissional
  \item Compilação em tempo real
  \item Gerenciamento de arquivos
  \item Gerenciamento de imagens
  \item Snippets prontos
  \item Linter LaTeX
  \item Assistente virtual
\end{itemize}

\section{Formatação de Texto}

Você pode usar \textbf{negrito}, \textit{itálico}, \underline{sublinhado} e \texttt{monoespaçado}.

\subsection{Listas Numeradas}

\begin{enumerate}
  \item Primeiro item
  \item Segundo item
  \item Terceiro item
\end{enumerate}

\section{Equações Matemáticas}

Equação inline: $E = mc^2$

Equação em destaque:
\begin{equation}
  \int_{a}^{b} f(x) \, dx = F(b) - F(a)
  \label{eq:fundamental}
\end{equation}

\section{Tabelas}

\begin{table}[h]
  \centering
  \begin{tabular}{|l|c|r|}
    \hline
    \textbf{Coluna 1} & \textbf{Coluna 2} & \textbf{Coluna 3} \\
    \hline
    Dado 1 & Dado 2 & Dado 3 \\
    Dado 4 & Dado 5 & Dado 6 \\
    \hline
  \end{tabular}
  \caption{Exemplo de tabela}
  \label{tab:exemplo}
\end{table}

\section{Imagens}

Para inserir imagens, use o botão verde "Imagem" na barra de ferramentas ou arraste a imagem para o editor.

% Exemplo de código para imagem (descomente quando tiver uma imagem)
% \begin{figure}[h]
%   \centering
%   \includegraphics[width=0.8\textwidth]{imagens/exemplo.png}
%   \caption{Legenda da imagem}
%   \label{fig:exemplo}
% \end{figure}

\section{Referências}

Você pode referenciar a equação~\ref{eq:fundamental} e a tabela~\ref{tab:exemplo}.

\section{Conclusão}

Este documento demonstra os recursos básicos do LaTeX. Experimente editar, adicionar imagens e compilar!

\end{document}


