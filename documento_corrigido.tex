\documentclass{article}

% Language setting
\usepackage[english]{babel}

% Set page size and margins
\usepackage[letterpaper,top=2cm,bottom=2cm,left=3cm,right=3cm,marginparwidth=1.75cm]{geometry}

% Useful packages
\usepackage{amsmath}
\usepackage{graphicx}

\title{Your Paper}
\author{You}

\begin{document}
\maketitle

\begin{abstract}
Your abstract.
\end{abstract}

\section{Introduction}

Your introduction goes here! Simply start writing your document and use the Recompile button to view the updated PDF preview. Examples of commonly used commands and features are listed below, to help you get started.

Once you're familiar with the editor, you can find various project settings in the DocCollab menu, accessed via the button in the very top left of the editor.

\section{Some examples to get started}

\subsection{How to create Sections and Subsections}

Simply use the section and subsection commands, as in this example document! With DocCollab, all the formatting and numbering is handled automatically according to the template you've chosen.

\subsection{How to include Figures}

First you have to upload the image file from your computer. Then use the includegraphics command to include it in your document. Use the figure environment and the caption command to add a number and a caption to your figure.

\subsection{How to add Tables}

Use the table and tabular environments for basic tables --- see Table~\ref{tab:widgets}, for example.

\begin{table}[h]
\centering
\begin{tabular}{l|r}
Item & Quantity \\\hline
Widgets & 42 \\
Gadgets & 13
\end{tabular}
\caption{\label{tab:widgets}An example table.}
\end{table}

\subsection{How to add Lists}

You can make lists with automatic numbering:

\begin{enumerate}
\item Like this,
\item and like this.
\end{enumerate}

Or bullet points:

\begin{itemize}
\item Like this,
\item and like this.
\end{itemize}

\subsection{How to write Mathematics}

\LaTeX{} is great at typesetting mathematics. Let $X_1, X_2, \ldots, X_n$ be a sequence of independent and identically distributed random variables with $\text{E}[X_i] = \mu$ and $\text{Var}[X_i] = \sigma^2 < \infty$, and let
\[S_n = \frac{X_1 + X_2 + \cdots + X_n}{n} = \frac{1}{n}\sum_{i}^{n} X_i\]
denote their mean. Then as $n$ approaches infinity, the random variables $\sqrt{n}(S_n - \mu)$ converge in distribution to a normal $\mathcal{N}(0, \sigma^2)$.

\subsection{Good luck!}

We hope you find DocCollab useful!

\end{document}

