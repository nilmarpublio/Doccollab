\documentclass[11pt,a4paper,twocolumn]{article}
\usepackage[utf8]{inputenc}
\usepackage[portuguese,brazil]{babel} 
\usepackage{amsmath}
\usepackage{amsfonts}
\usepackage{amssymb}
\usepackage{graphicx}

\usepackage{geometry}
\geometry{a4paper,left=1.8cm,right=1.8cm,top=2cm,bottom=1.5cm}

\author{Leandro Camara Ledel}
\title{Artigo de teste de edição de textos com o LaTeX}

\begin{document}

\maketitle

\section*{Resumo}
O presente artigo é um exercício de introdução ao LaTex. Selecionamos alguns ítens de organização dos assuntos, como elementos de criação de novas partes (como é comum em livros), de novas seções (como em artigos e teses), e de inserção de novos parágrafos.

\section*{Agradecimentos}
Agradecemos em especial aos nossos dois ilustres companheiros de laboratório, detentores de avançados conhecimentos do LaTeX, cujos nomes serão mantidos no mais absoluto sigilo, em virtude da possível exposição pública do presente documento.

\section{Introdução}
O presente artigo objetiva experimentar alguns dos recursos básicos do LaTeX, afim de compor um pequeno guia introdutório para usuários iniciantes do LaTex.

\section{Testes de Formatação de Parágrafos}
Vejamos como se faz para digitar um texto em duas colunas. Muito bem, de fato funciona, entretanto agora surge a dúvida de como fazer para inserir um título que utilize as duas colunas.

O próximo passo é descobrir como se faz para alterar o título criado.

\section{Configuração da Página}
Descobriu-se que as margens estavam muito grandes. Corrigiu-se este problema através do pacote geometry.
 
\section{Inserção de Figuras}
A inserção de figuras pode ser feita utilizando uma ferramenta de desenho e produzindo um arquivo no formato adequado.

\section{Conclusão}
O trabalho com o LaTex certamente é compensador, em virtude da excelente qualidade dos documentos produzidos.

\end{document}\end{document}